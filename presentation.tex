\chapter{Présentation de l'entreprise}

\section{L'entreprise \asensio}

La société \asensio\ a été créée en 1988 par les deux co-fondateurs \apotencier\ et \apascal. Dès ses premiers mois, elle s'est très vite orientée vers les technologies de l'\ainternet. Elle s'inscrit alors comme une véritable agence web interactive, proposant à ses clients un savoir-faire dans tous les métiers du web : développement, expertise technique, \awm, communication et \awd.

En 2007, \asensio\ s'est rapprochée d'\aextreme\footnote{À l'occasion du 29ème Grand Prix des \aagencesannee en 2009, \aextreme s'est vu décerné le prix du groupe de communication indépendant de l'année}, un grand groupe indépendant de communication globale\footnote{publicité, marketing, services, web, \textit{packaging}, design, \textit{corporate}}. Cette démarche commerciale, et non capitalistique, distingue bien les deux entités en deux sociétés à part entière. Elle a pour origine un désir des deux parties : d'un côté \aextreme\ souhaitait se rapprocher d'une agence web, et de l'autre, \asensio\ ressentait le besoin d'acquérir de meilleures compétences dans le domaine de la communication.

L'association de \asensio\ et d'\aextreme\ a donné naissance à trois entités commerciales, ou \abusfull\ (\abus) :

\begin{description}
	\item[\asl] s'occupe de la partie développement web ;
	\item[\aes] gère tout l'aspect \awm\ et communication ;
	\item[\aesm] propose à ses clients des plans média destinés à doper leur trafic et leurs ventes. 
\end{description}

Rentable dès ses débuts, \asensio\ dégage un chiffre d'affaires de 6~millions d'euros pour l'année~2009, d'après les estimations actuelles.


\section{La \abu\ \asl}

TODO


\section{Mon choix de \asl}

TODO
