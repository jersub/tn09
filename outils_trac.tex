\subsection{\atrac}
\label{section:outils_trac}

\atrac\footnote{\url{http://trac.edgewall.org/}} est un logiciel libre de gestion de projet, prenant la forme d'une interface web. Il inclut notamment un wiki\footnote{Un wiki est un site web dont les pages sont généralement modifiables par l'ensemble des visiteurs du site. Il permet ainsi l'écriture collaborative de documents.~\cite{wiki}} pour partager du contenu, un système de suivi de tickets\footnote{Un ticket comprend généralement un titre, un contenu, et différents statuts de suivi, comme \og nouveau \fg\ ou \og résolu \fg\ par exemple. Un ticket est souvent utilisé pour représenter un rapport de \abug.}, un explorateur de dépôt \asvn\ et une vue d'historique.

Chez \asl, chaque projet possède son propre \atrac, qui facilite sa gestion et son suivi. Ce dernier est utilisé en tant que moyen de communication privilégié entre développeurs, chefs de projet et clients. Il prend tout son intérêt lors des phases de recette, comme on le verra en section~\ref{section:eyrolles_organisation_recette}.
