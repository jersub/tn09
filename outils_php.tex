\subsection{Le langage \aphp}

\aphp\footnote{\url{http://www.php.net}} est le langage de programmation utilisé chez \asl. C'est un langage de script libre principalement utilisé pour produire des pages web dynamiques : le code \aphp\ est intégré dans un document \ahtml\ puis interprété par le serveur web doté d'un module \aphp. Le langage peut également être utilisé via un interpréteur lancé via la ligne de commande, pour exécuter des programmes localement par exemple.

Sa syntaxe est empruntée aux langages C, Java et Perl, afin de rendre le langage plus facile à apprendre. Depuis la version~5, \aphp\ est capable de produire des programmes à la conception orientée objet.
